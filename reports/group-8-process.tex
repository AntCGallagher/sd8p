\documentclass[a4paper,12pt]{article}
\usepackage[final]{graphicx}
\graphicspath{{pics/}}
\usepackage{float}
\usepackage{hyperref}
\usepackage{subfig}

\begin{document}
\begin{titlepage}
	\centering
	{\scshape\LARGE The University of Edinburgh \par}
	\vspace{1cm}
	{\scshape\Large System Design Project\par}
	\vspace{1.5cm}
	{\huge\bfseries Process Report\par}
	\vspace{2cm}
	{\Large\itshape Group 8\par}
	\vfill
\begin{figure}[H]
    \centering
    \includegraphics[width=10cm]{robot}
    \label{fig:robot}
\end{figure}
	\vfill

	{\large February 27, 2017\par}
\end{titlepage}

\section{Introduction}
This document is the Process Report for the Group SD-8P (Group 8). It includes an overview of how the group is structured, how it communicates and how it formulated plans for this particular group project.
\subsection{Tasks}
The first order of business was to split this colossal task of getting a robot to play football into a smaller subset of tasks. As a team, there was a discussion to determine what different tasks there were and, following that discussion, five main tasks were created. The group then proceeded to split the tasks by simply asking who wanted to do which task. The teams originally assigned to each task were as follow:

\begin{figure}[H]
    \centering
    \includegraphics[width=10cm]{roles}
    \caption{Diagram with Tasks Assigned}
    \label{fig:roles}
\end{figure}

\subparagraph{Team Manager}
Andra took the role of lead Team Manager while Filipe would help in anything that was needed. The Team Managers would be responsible for ensuring appropriate communication with the mentor. Examples of other task that were part of the Team Managers responsibilities are: organizing weekly meetings, making sure everyone is doing an equal amount of work and keeping track of everyone's work.


\subparagraph{Vision}
Alex and Vesko both volunteered to do the Vision part of the assignment. The Vision team would be responsible for building a world model using the camera feed and passing that to the Strategy team. Examples of tasks that the Vision team had to accomplish for are: calibrating the camera feed from the pitch rooms, manipulating the images coming from the camera feed and analysing the processed data from the camera feed.

\subparagraph{Strategy}
Danial and Filipe decided to do the Strategy part of the assignment. The Strategy team would be responsible for building a strategy that the robot can use during friendlies and the final day. Examples of tasks that the Strategy team would have to do are: communicating with the different teams to understand how each part of the project is going on, developing a strategy for the robot and coordinating with the other group to further progress both robots.

\subparagraph{Communications}
Anthony chose to do the Communications part of the assignment. The Communications team would be responsible for transferring messages between the computer and the Arduino using the RF stick. This task involved many smaller task such as: testing the RF stick, implementing a protocol to send/receive messages and ensuring the messages are being processed correctly on the robot.

\subparagraph{Hardware}
Andra, Lazar and Stefani decided to do the Hardware part of the assignment. The Hardware team would be responsible to design and build the robot. They would also have to: implement the different instructions that the robot would have to follow, think of a kicker/grabber mechanism and find the needed parts to build the robot.
\\ \\
There have been no movements between teams but Stefani is no longer taking SDP so the hardware team has been reduced to two members. At many times during the project the different teams also help each other in their respective parts (so no singular team is working only with themselves).

\section{Communication Channels}
Communication is key when undertaking a project of this size. At the first meeting it was decided that we would meet at least once a week to discuss how everyone is coming along in their roles. We set up the communication channels as such:

\subparagraph{Facebook Messenger}
Most of our information would be sent through the Facebook Messenger. It is by far the most convenient channel of communication and the one that people check the most (so any messages sent here would be hard to miss). We also communicate with the other group through this channel of communication.
\subparagraph{Slack}
Any formal communications between groups, questions to the mentors or talking to our own mentor would be done through Slack. Slack is the only platform where all the participants/organizers of SDP are, so any questions regarding SDP would be posted there.
\subparagraph{Asana}
Although Facebook Messenger was our primary communications channel, we still used Asana to monitor and track some goals. Each group had their own agenda leading up to the different deadlines and each member could check Asana to see what another group is up to.
\subparagraph{Team Meetings}
Generally whenever a team had to work on certain tasks they would meet together. In these meetings they would work together in their daily goals and would share information between the present members of the team.

\section{Task Allocation}
The majority of the tasks allocated were done by the individual subgroups and not as a centralized unit. The individual teams mostly come up with daily tasks as they seemed fit (this will change after the second friendly as from then on we will start working with the other group and we are planning to set goals together). 

\section{Progress Tracking}
\subparagraph{Asana}
We tracked the progress of the team with Asana. We used its functionality to create tasks and assign them to ourselves or one another; add due date; split the tasks across parts of the project; and mark tasks as complete. As complete tasks would still be visible, we are able to track previous progress and plan our future tasks. We decided to split the project into Hardware, Strategy and Vision.

\subparagraph{Planning}
At the onset of SDP, we held a meeting where we identified the subtasks for each part of the project. We used those subtasks to create a work breakdown structure (see Task Allocation figure). The structure would be later edited with new or modified subtasks.
\\ \\
In the beginning of the day people who work together set goals to complete during that day. At the end of the day they would assess their progress and discuss a plan for their next meeting. This gave us a clear, coherent idea of which tasks have been completed that day and which task to tackle next. We update Asana and the work breakdown structure to show the current progress.

\subparagraph{Comparing}
We spend some time assessing the functionality of other team's robots. This gives us a realistic idea of where we stand among all teams and the opportunity to learn new techniques.

\subparagraph{Milestones}
We created a Gantt chart to depict the constraints between the subtasks and estimate the completion time for our milestones. Thus, we can easily separate parallel and serial subtasks and identify slack time. Our aim has been to stick to the Gantt chart schedule and reassess it when there are substantial variations. While each subtask is a minor milestone, we have been working towards getting optimal performance for each of the matches.

\begin{figure}[H]
    \centering
    \includegraphics[width=15cm]{WorkSDP}
    \label{fig:robot}
\end{figure}
	\vfill

\section{Risks}

\subsection{Potential Risks}
\input{potential}

\subsection{Risk Assessment}
\subparagraph{Code loss/GitHub}
Pieces of code can be lost due to deletion or modification. They can also be corrupted due to a recent change. We use GitHub because it allows us to go back in time to restore the previous state of our lost/corrupt code or just observe the differences to manually fix a issue.

\subparagraph{Teammate absence/User Guide}
Some members of our team may be absent for long periods of time due to unforeseen circumstances. Their absence may cause a halt of the progress of the production process if an error has occurred. To safeguard, we have created a user guide (i.e. README) that would provide a brief description of the developed system and  instructions for solving common errors. On the day of the match we have at least one additional member present to fill in for an absent operator or handler.  Our project managers track everyone's progress and are able to perform the task of any missing teammate. 

\subparagraph{Issues on match day}
Our main battery pack may discharge faster than expected on match day. We keep a second battery pack charged for emergencies on match day. In case our computer in the pitch room is taken by the other team for calibration just before the match, we perform consecutive calibrations before the match and use the most recent one.

\subparagraph{Late Delivery}
In the case of late delivery of one of system's main parts, we would use the last known functional version of that part. We look to minimise the impact late delivery would cause on the other teammates and the part they develop.

\subparagraph{Task specific risk assessment}
Each group member has recorded a personal risk assessment and contingency plan in the group's log. Members from different subgroups mentioned different risks. A member of the hardware team, for example,   considered some potential risks like the speed of the robot being too slow or cases where the robot does not respond to any message. They then developed a contingency plan around that. 
\\ \\ The log provides a way for each member to assess the risks associated with their tasks and also allows the other members of the team to understand what everyone is up to.

\iffalse
key is hidden in safe place
have 2 code repos we have run - fred and craig's robot
what to do in budget inefficiency?
how to test if pitch is full?
IDEAS
Where should staff meet in the event the building is not accessible?
Who has the authority to close the business in the event of an emergency?
Which staff members are critical and must be on-site or always reachable?
Where are the back-ups and how are they restored?
Who can cover for each critical staff member?
Who are single points of failure and how can those risks be ameliorated?
What systems, vendors, and partners pose risk should they fail?
Who is responsible for communicating with customers, and how?
\fi

\subsection{Contingency Planning}
\input{contingency}

\section{Conclusion}
In face of such a large task the group has kept itself on track in an organized, and relaxed, fashion. Expectations were high for the first friendly, but due to several different factors the robot did not perform as expected. We are currently awaiting to see the results for the second friendly (also before this report is due).

\end{document}
