\subparagraph{Asana}
We tracked the progress of the team with Asana. We used its functionality to create tasks and assign them to ourselves or one another; add due dates; split the tasks across parts of the project; and mark tasks as completed. As completed tasks would still be visible, we are able to track previous progress and plan our future tasks. We decided to split the project, in Asana, into Communications, Hardware, Strategy, Vision and Team Meetings (for the Team Managers).

\subparagraph{Planning}
At the onset of SDP, we held a meeting where we identified the subtasks for each part of the project. We used those subtasks to create a Work Breakdown Structure diagram (refer to Figure \ref{fig:diagram}). The structure would be later edited with new or modified subtasks.

\begin{figure}[H]
    \centering
    \includegraphics[width=15cm]{WorkSDP}
    \caption{Work Breakdown Structure diagram.}
    \label{fig:diagram}
\end{figure}

In the beginning of a working session, people who work together set goals to complete. At the end of the session, they would assess their progress and discuss a plan for their next meeting. This gave us a clear idea of which tasks have been completed in a given section and which tasks to tackle next. We update Asana and the Work Breakdown Structure to show the current progress.

\subparagraph{Comparing}
We spend some time assessing the functionality of other teams' robots. This gives us a realistic idea of where we stand among all teams and the opportunity to learn new techniques.

\subparagraph{Milestones}
We created a Gantt chart (refer to Appendix C) to depict the constraints between the subtasks and estimate the completion time for our milestones. Thus, we can easily separate parallel and serial subtasks and identify slack time. Our aim has been to stick to the Gantt chart schedule and reassess it when there are substantial variations. While each subtask is a minor milestone, we have been working towards getting optimal performance for each of the matches.