Communication is key when undertaking a project of this size. At the first meeting it was decided that we would meet at least once a week to discuss how everyone is coming along in their roles. We set up the communication channels as such:

\subparagraph{Facebook Messenger}
Most of our information would be sent through the Facebook Messenger. It is by far the most convenient channel of communication and the one that people check the most (so any messages sent here would be hard to miss). We also communicate with the other group through this channel of communication.
\subparagraph{Slack}
Any formal communications between subgroups, questions to the mentors or talking to our own mentor would be done through Slack. Slack is the only platform where all the participants/organizers of SDP are, so any questions regarding SDP would be posted there.
\subparagraph{Asana}
Asana is a web application designed to help a team track their work. It visualises todo tasks with deadlines as a list or as a calendar.
We used Asana to monitor and track various goals. Each subgroup had their own agenda leading up to the different deadlines and each member could check Asana to see what another subgroup was up to.
\subparagraph{Team Gatherings}
Generally whenever a team had to work on certain tasks they would meet together. In these gatherings they would work together in their daily goals and would share information between the present members of the team. There were also weekly meetings where the team would meet with the mentor.
\subparagraph{GitHub}
The team's GitHub repository also serves as a channel of communication since people write a message describing what they did when they commit to the repository. This way, by checking the commit history, one can have an idea on what the different teams were doing on a particular day.
