\subparagraph{Asana}
We tracked the progress of the team with Asana. We used its functionality to create tasks and assign them to ourselves or one another; add due date; split the tasks across parts of the project; and mark tasks as complete. As complete tasks would still be visible, we are able to track previous progress and plan our future tasks. We decided to split the project into Hardware, Strategy and Vision.

\subparagraph{Planning}
At the onset of SDP, we held a meeting where we identified the subtasks for each part of the project. We used those subtasks to create a work breakdown structure (see Task Allocation figure). The structure would be later edited with new or modified subtasks.
\\ \\
In the beginning of the day people who work together set goals to complete during that day. At the end of the day they would assess their progress and discuss a plan for their next meeting. This gave us a clear, coherent idea of which tasks have been completed that day and which task to tackle next. We update Asana and the work breakdown structure to show the current progress.

\subparagraph{Comparing}
We spend some time assessing the functionality of other team's robots. This gives us a realistic idea of where we stand among all teams and the opportunity to learn new techniques.

\subparagraph{Milestones}
We created a Gantt chart to depict the constraints between the subtasks and estimate the completion time for our milestones. Thus, we can easily separate parallel and serial subtasks and identify slack time. Our aim has been to stick to the Gantt chart schedule and reassess it when there are substantial variations. While each subtask is a minor milestone, we have been working towards getting optimal performance for each of the matches.