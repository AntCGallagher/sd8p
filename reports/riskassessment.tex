\subparagraph{Code loss/Git}
Pieces of code can be lost due to deletion or modification. They can also be corrupted due to a recent change. We use Git because it allows us to go back in time to restore the previous state of our lost/corrupt code or just observe the differences to manually fix a issue.

\subparagraph{Teammate absence/User Guide}
Some members of our team may be absent for long periods of time due to unforeseen circumstances. Their absence may cause a halt of the progress of the production process if an error has occurred. To safeguard against this, we have created a user guide (i.e. README) that would provide a brief description of the developed system and  instructions for solving common errors. On the day of the match we have at least one additional member present to fill in for an absent operator or handler.  Our project managers track everyone's progress and are able to perform the task of any missing teammate. 

\subparagraph{Issues on match day}
Our main battery pack may discharge faster than expected on match day. We keep a second battery pack charged for emergencies on match day.

\subparagraph{Late Delivery}
In the case of late delivery of one of system's main parts, we would use the last known functional version of that part. We would achieve that by reverting to the last working commit in GitHub. Thus, we look to minimise the impact late delivery would cause on the other teammates and the part they develop.

\subparagraph{Task specific risk assessment}
Each group member, periodically, has recorded a personal risk assessment and contingency plan in the group's log by answering some pre-defined questions (refer to the appendices for excerpts of the group's log). Members from different subgroups mentioned different risks. A member of the hardware team, for example,   considered some potential risks like the speed of the robot being too slow or cases where the robot does not respond to any message. They then developed a contingency plan around that. 
\\ \\ The log provides a way for each member to assess the risks associated with their tasks and also allows others to check on the rest of the group members.

\iffalse
key is hidden in safe place
have 2 code repos we have run - fred and craig's robot
what to do in budget inefficiency?
how to test if pitch is full?
IDEAS
Where should staff meet in the event the building is not accessible?
Who has the authority to close the business in the event of an emergency?
Which staff members are critical and must be on-site or always reachable?
Where are the back-ups and how are they restored?
Who can cover for each critical staff member?
Who are single points of failure and how can those risks be ameliorated?
What systems, vendors, and partners pose risk should they fail?
Who is responsible for communicating with customers, and how?
\fi
